\section*{ЗАКЛЮЧЕНИЕ}
\addcontentsline{toc}{section}{ЗАКЛЮЧЕНИЕ}

В данной выпускной квалификационной работе была разработана распределенная поисковая система для Интернета. Данный проект направлен на эффективное ориентирование в информации из WWW.

Основные результаты работы:

\begin{enumerate}
	\item Проведен анализ предметной области и было изучено устройство современных поисковых систем. На его основе были выделены такие компоненты, как поисковый робот, индексатор и поисковый интерфейс.
	\item Разработана концептуальная модель распределенной поисковой системы.
	\item Спроектированы все компоненты распределенной поисковой системы: поисковый робот, индексатор, поисковый интерфейс и сборщик журналируемой информации. Их взаимодействие осуществляется с помощью брокера сообщений RabbitMQ.
	\item Каждый из компонентов распределенной поисковой системы был реализован в соответствие с предъявляемыми ему функциональными требованиями. Поисковый робот занимается обходом документов из Интернета, индексатор -- формированием индекса для эффективного поиска среди большого множества документов, поисковый интерфейс -- поиском наиболее релевантных документов по запросу с помощью сформированного индекса, а сборщик журналируемой информации помогает сохранять данные о работе всех компонентов для будущей диагностики. 
	\item Спроектирован и реализован WEB-интерфейс для поисковой системы. Данный компонент предназначен для отображания результатов поиска. 
\end{enumerate}

Все требования, указанные в техническом задании, были полностью реализованы. Поставленные задачи в начале разработки проекта были успешно решены.

Распределенная поисковая система для Интернета предоставляет возможность получать наиболее релевантные документы по пользовательскому запросу.