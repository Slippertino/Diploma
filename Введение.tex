\section*{ВВЕДЕНИЕ}
\addcontentsline{toc}{section}{ВВЕДЕНИЕ}

Интернет -- всемирная система объединенных компьютерных сетей для хранения и передачи информации, построенная на стеке протоколов TCP/IP. На его основе работает Всемирная паутина (WWW,  web) -- распределенная система, предоставляющая доступ к связанным между собой документам.  Большинство ресурсов Всемирной паутины основано на технологии гипертекста. Гипертекстовые документы, размещаемые во Всемирной паутине, называются веб-страницами.

Веб-страницы часто содержат очень полезную для той или иной деятельности информацию, но ввиду огромного их количества возникает проблема эффективности ориентирования между ними. Поэтому из нее вполне закономерно следует задача о возможности подбора наиболее подходящих страниц, где фильтром будет выступать пользовательский текстовый запрос. Именно эту задачу и решают поисковые системы.

\emph{Цель настоящей работы} – разработка распределенной поисковой системы для Интернета. Для достижения поставленной цели необходимо решить \emph{следующие задачи:}
\begin{itemize}
\item провести анализ предметной области;
\item разработать концептуальную модель распределенной поисковой системы;
\item спроектировать компоненты распределенной поисковой системы и их взаимодействие между собой;
\item реализовать каждый из компонентов распределенной поисковой системы;
\item разработать поисковый web-сайт.
\end{itemize}

\emph{Структура и объем работы.} Отчет состоит из введения, 4 разделов основной части, заключения, списка использованных источников, 2 приложений. Текст выпускной квалификационной работы равен \formbytotal{lastpage}{страниц}{е}{ам}{ам}.

\emph{Во введении} сформулирована цель работы, поставлены задачи разработки, описана структура работы, приведено краткое содержание каждого из разделов.

\emph{В первом разделе} на стадии описания технической характеристики предметной области приводится сбор информации об устройстве работы современных поисковых систем.

\emph{Во втором разделе} на стадии технического задания приводятся требования к распределенной поисковой системе.

\emph{В третьем разделе} на стадии технического проектирования представлены проектные решения для распределенной поисковой системы.

\emph{В четвертом разделе} приводится список классов и их методов, использованных при разработке распределенной поисковой системы, производится её системное тестирование.

В заключении излагаются основные результаты работы, полученные в ходе разработки.

В приложении А представлен графический материал.
В приложении Б представлены фрагменты исходного кода. 