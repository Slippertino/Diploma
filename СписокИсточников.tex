\addcontentsline{toc}{section}{СПИСОК ИСПОЛЬЗОВАННЫХ ИСТОЧНИКОВ}

%\begin{hyphenrules}{nohyphenation} %отключение переноса слов в содержании
\begin{thebibliography}{11}
	\bibitem{ir} Information retrieval : implementing and evaluating search engines / Stefan Buttcher, Charles L.A. Clarke, and Gordon V. Cormack. ~- Library of Congress Cataloging-in-Publication Data. ~- 624 c.~- ISBN 978-0-262-02651-2 ~– Текст~: непосредственный.
	\bibitem{architecture} Клеппман, М. Высоконагруженные приложения. Программирование, масштабирование, поддержка~/ М. Клеппман.~– Санкт-Петербург~: Питер, 2018.~– 640 с.~– ISBN 978-5-44-610512-0.~– Текст~: непосредственный.
	\bibitem{cpp} Страуструп, Б. Язык программирования C++: Специальное издание / Б. Страуструп; Пер. с англ. Н.Н. Мартынов. — М.: БИНОМ, 2017. — 1136 c ~- ISBN: 5-7989-0223-4.~- Текст~: непосредственный.
	\bibitem{cppconcurrent} Энтони Уильямс. Параллельное программирование на C++ в действии. Практика разработки многопоточных программ. Пер. с англ. Слинкин А. А. ~– М.: ДМК Пресс, 2012. ~– 672с.: ил.ISBN 978-5-94074-448-1. ~- Текст~: непосредственный.
	\bibitem{clearCode}	Мартин, Р. Чистый код. Создание, анализ и рефакторинг~/ Р. Мартин.~– Санкт-Петербург~: Питер, 2020.~– 464 с.~– ISBN 978-5-4461-0960-9.~– Текст~: непосредственный.
    \bibitem{architecture1} Баланов, А. Построение микросервисной архитектуры и разработка высоконагруженных приложений. Учебное пособие~/ А. Баланов.~– Москва~: Лань, 2024.~– 244 с.~– ISBN 978-5-507-48747-9.~– Текст~: непосредственный.
	\bibitem{architecture2} Фримен, Э. Паттерны проектирования. Обновленное юбилей-ное издание. / Э. Фримен, Э. Робсон, К. Сьерра, Б. Бейтс – СПБ.: Питер, 2018. – 656с. – ISBN 978–5–496–03210–0. – Текст : непосредственный.
	\bibitem{clearCode}	Webber, J. REST in Practice. Hypermedia and Systems Architecture~/ J. Webber.~– Санкт-Петербург~: Питер, 2010.~– 448 с.~– ISBN 978-0-5968-0582-1.~– Текст~: непосредственный.
	\bibitem{algorinms} Бхаргава, А. Грокаем алгоритмы. Иллюстрированное пособие для программистов и любопытствующих~/ А. Бхаргава.~– Санкт-Петербург~: Питер, 2022~– 288 с.~– ISBN 978-5-4461-0923-4.~– Текст~: непосредственный.
	\bibitem{postgres} Джуба С., Волков А. Изучаем PostgreSQL 10 / пер. с анг. А. А. Слинкина. ~– М.: ДМК Пресс, 2019. ~– 400 с.: ил. ISBN 978-5-97060-643-8.~– Текст~: непосредственный.
	\bibitem{rabbitmq} RabbitMQ : RabbitMQ  documentation : сайт. - URL: https://www.rabbitmq.com/docs (дата обращения: 17.04.2024).~– Текст~: электронный.
\end{thebibliography}
%\end{hyphenrules}