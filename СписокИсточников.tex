\addcontentsline{toc}{section}{СПИСОК ИСПОЛЬЗОВАННЫХ ИСТОЧНИКОВ}

%\begin{hyphenrules}{nohyphenation} %отключение переноса слов в содержании
\begin{thebibliography}{51}
	\bibitem{ir} Маннинr, К. Д. Введение в информационный поиск / К. Д. Маннинг, П. Рагхаван, Х. Шюнце; Пер. с англ. - Санкт-Петербург : ООО "<Даиалектика">, 2020. - 528 c. - ISBN  978-5-907203-20-4. - Текст : непосредственный.
	\bibitem{ir2} Колисниченко, Д. Н. Поисковые системы и продвижение сайтов в Интернете / Д. Н. Колисниченко. — Москва : Диалектика, 2007. — 270 c. — ISBN 978-5-8459-1269-5. – Текст : непосредственный.
	\bibitem{ir3} Ландэ, Д. В. Навигация в сложных сетях: модели и алгоритмы / Ландэ Д. В., Снарский А. А., Безсуднов И. В. - Москва : Либроком, 2009. - 264 с. - ISBN 978-5-397-00497-8. - Текст : непосредственный.
	\bibitem{arch2} Ричардc, М. Фундаментальный подход к программной архитектуре: паттерны, свойства, проверенные методы / М. Ричардc, Н. Форд. — Санкт-Петербург : Питер, 2023. — 448 с. - ISBN 978-5-4461-1842-7. – Текст : непосредственный.
	\bibitem{uml1} Фаулер, М. UML. Основы: краткое руководство по стандартному языку объектного моделирования / М. Фаулер. – Символ-Плюс, 2011. – 184 с. – ISBN 978-5-93286-060-1. – Текст : непосредственный.
    \bibitem{arch1} Баланов, А. Построение микросервисной архитектуры и разработка высоконагруженных приложений. Учебное пособие / А. Баланов. – Москва : Лань, 2024.– 244 с. – ISBN 978-5-507-48747-9. – Текст : непосредственный.
	\bibitem{cpp1} Страуструп, Б. Язык программирования C++: Специальное издание / Б. Страуструп; Пер. с англ. Н.Н. Мартынов. — Москва : БИНОМ, 2017. — 1136 c. - ISBN: 5-7989-0223-4. - Текст : непосредственный.
	\bibitem{cpp2} Уильямс, Э. Параллельное программирование на C++ в действии. Практика разработки многопоточных программ / Э. Уильямс; Пер. с англ. Слинкин А. А. – Москва : ДМК Пресс, 2012. – 672с. - ISBN 978-5-94074-448-1. - Текст : непосредственный.
	\bibitem{js}  Фримен, Э. Изучаем программирование на JavaScript / Э. Фримен, Э. Робсон. – Санкт-Петербург : Питер, 2018. – 640 с. – ISBN 978-5-4461-0893-0. – Текст : непосредственный.	
	\bibitem{cmake} Дубров, Д. В. Система построения проектов CMake / Д. В. Дубров. — Ростов-на-Дону : Издательство Южного федерального университета, 2015. — 419 с. - ISBN 978-5-9275-1852-4. – Текст : непосредственный.
	\bibitem{vcpkg} Vcpkg : Документация по vcpkg : cайт. - URL: https://learn.microsoft.com/ru-ru/vcpkg (дата обращения: 13.03.2024). – Текст : электронный.
	\bibitem{postman} Postman : Документация по Postman : сайт. - URL: https://learning.postman.com/docs (дата обращения: 29.04.2024). - Текст : электронный.
	\bibitem{postgres1} PostgreSQL : Документация по PostgreSQL : сайт. - URL: https://www.postgresql.org/docs (дата обращения: 01.04.2024). - Текст : электронный.
	\bibitem{postgres2} Джуба, С. Изучаем PostgreSQL 10 / С. Джуба, А. Волков; пер. с анг. А. А. Слинкина. – Москва : ДМК Пресс, 2019. – 400 с. - ISBN 978-5-97060-643-8. – Текст : непосредственный.
	\bibitem{amqp} AMQP : Документация по AMQP : сайт. - URL: https://www.rabbitmq.com/resources/specs/amqp0-9-1.pdf (дата обращения: 22.04.2024). – Текст : электронный.
	\bibitem{rabbitmq} RabbitMQ : Документация по RabbitMQ : сайт. - URL: https://www.rabbitmq.com/docs (дата обращения: 03.04.2024). – Текст : электронный.
	\bibitem{userver} Userver : Документация по Userver : сайт. - URL: https://userver.tech/de/d6a/md\_en\_2index.html (дата обращения: 20.04.2024). – Текст : электронный.
	\bibitem{boost} Boost : Документация по Boost : сайт. - URL: https://www.boost.org/doc/libs/1\_84\_0 (дата обращения: 05.03.2024). – Текст : электронный.
	\bibitem{cds} LibCDS : Документация по LibCDS : сайт. - URL: https://libcds.sourceforge.net/doc (дата обращения: 07.04.2024). – Текст : электронный.
	\bibitem{tbb} IntelTBB : Документация по IntelTBB : сайт. - URL: https://oneapi-src.github.io/oneTBB (дата обращения: 15.04.2024). – Текст : электронный.	
	\bibitem{pqxx} Libpqxx : Документация по Libpqxx : сайт. - URL: https://libpqxx.readthedocs.io/en/7.8.0/index.html (дата обращения: 01.04.2024). – Текст : электронный.
	\bibitem{amqpcpp} AMQP-CPP : Документация по AMQP-CPP : сайт. - URL: https://github.com/CopernicaMarketingSoftware/AMQP-CPP (дата обращения: 02.04.2024). – Текст : электронный.
	\bibitem{yamlcpp} Yaml-Cpp : Документация по Yaml-Cpp : сайт. - URL: https://github.com/jbeder/yaml-cpp/blob/master/docs (дата обращения: 02.04.2024). – Текст : электронный.
	\bibitem{cld3} Cld3 : Документация по Cld3 : сайт. - URL: https://github.com/google/cld3 (дата обращения: 19.03.2024). – Текст : электронный.
	\bibitem{libarchive} Libarchive : Документация по Libarchive : сайт. - URL: https://github.com/libarchive/libarchive/wiki (дата обращения: 07.05.2024). – Текст : электронный.
	\bibitem{gumbo} Gumbo : Документация по Gumbo : сайт. - URL: https://github.com/google/gumbo-parser/tree/master/examples (дата обращения: 13.04.2024). – Текст : электронный.
	\bibitem{libstemmer} Libstemmer : Документация по Libstemmer : сайт. - URL: https://github.com/zvelo/libstemmer (дата обращения: 17.04.2024). – Текст : электронный.
	\bibitem{robotstxt} Robotstxt : Документация по Robotstxt : сайт. - URL: https://github.com/google/robotstxt (дата обращения: 14.04.2024). – Текст : электронный.
	\bibitem{quill} Quill : Документация по Quill : сайт. - URL: https://quillcpp.readthedocs.io/en/latest/ (дата обращения: 12.04.2024). – Текст : электронный.
	\bibitem{uml2}  Буч, Г. Введение в UML от создателей языка / Г. Буч, И. Якобсон, Д. Рамбо. – Москва : ДМК Пресс, 2015. – 498 с. – ISBN 978-5-457-43379-3. – Текст : непосредственный.
	\bibitem{robots} Robots Exclusion Protocol : Спецификация протокола REP : сайт. - URL: https://datatracker.ietf.org/doc/rfc9309 (дата обращения: 15.04.2024). - Текст : электронный.
	\bibitem{sitemap} Sitemaps : Спецификация протокола Sitemaps : cайт. - URL: https://www.sitemaps.org/protocol.html (дата обращения: 15.04.2024). - Текст : электронный.
	\bibitem{db1} Осипов, Д. Технологии проектирования баз данных / Д. Осипов. – Москва : ДМК-Пресс, 2019. – 498 с. – ISBN 978-5-9706-0737-4. – Текст : непосредственный.
	\bibitem{oop1} Паттерны объектно-ориентированного проектирования / Э. Гамма, Р. Хелм, Р. Джонсон, Дж. Влиссидес. – Санкт-Петербург : Питер, 2020. – 448 с. – ISBN 978-5-4461-1595-2. – Текст : непосредственный.
	\bibitem{algo1} Кормен, Т. Х. Алгоритмы: построение и анализ, 2-е издание / Т. Х. Кормен, Ч. И. Лейзерсон, Р. Л. Ривест, К. Штайн. - Москва : Издательский дом "<Вильямс">, 2011. - 1296 с. - ISBN 978-5–8459–0857-5. - Текст : непосредственный.
	\bibitem{math1} Кельберт, М. Я. Вероятность и статистика в примерах и задачах. Т. ІІ: Марковские цепи как отправная точка теории случайных процессов и их приложения / М. Я. Кельберт, Ю. М. Сухов  — Москва : МЦНМО, 2010. — 295 с. — ISBN 978-5-94057-252-7. – Текст : непосредственный.
	\bibitem{math2} Гихман, И. И. Теория вероятностей и математическая статистика: учебник для мат. спец. ун-тов и техн. вузов / И. И. Гихман, А. В. Скороход, М. И. Ядренко. — 2-е изд., перераб. и доп. — Киев: Выща шк., 1988. — 439 с. — ISBN 5-11-000108-1. – Текст : непосредственный.
	\bibitem{algo2} Ахо, А. Структуры данных и алгоритмы / А. Ахо. – Вильямс, 2010. – 391 с. – ISBN 978-5-8459-1610-5. – Текст : непосредственный.
	\bibitem{db2} Дейт, К. Дж. Введение в системы баз данных, 8-е издание / К. Дейт. — Москва : Издательский дом "Вильяме", 2005. — 1328 с. - ISBN 5-8459-0788-8. – Текст : непосредственный.
	\bibitem{pi1} Лаврищева, Е. Программная инженерия и технологии программирования сложных систем / Е. Лаврищева. – 2-е изд., испр. и доп. – Москва : Юрайт, 2018. ~– 432 с. ~– ISBN 978-5-04-315935-9. ~– Текст : непосредственный.
	\bibitem{algo3} Бхаргава, А. Грокаем алгоритмы. Иллюстрированное пособие для программистов и любопытствующих / А. Бхаргава. – Санкт-Петербург : Питер, 2022 – 288 с. – ISBN 978-5-4461-0923-4. – Текст : непосредственный.
	\bibitem{pi2} Орлов, С. А. Программная инженерия. Учебник для вузов / С. А. Орлов. – 5-е изд., обновленное и дополненное. – Санкт-Петербург : Питер, 2021. – 640 с. – ISBN 978-5-4461-9590-9. – Текст : непосредственный.
	\bibitem{oop2} Зайцев, М. Г. Объектно-ориентированный анализ и программирование / М. Г. Зайцев. – Новосибирск : изд-во НГТУ, 2017. – 84 с. – ISBN 978-5-04-112962-0. – Текст : непосредственный.
	\bibitem{web}  Биэль, М. RESTful API Design / М. Биэль. – University Press, 2016. – 300 с. – ISBN 978-1-5147-3516-9. – Текст : непосредственный.
	\bibitem{clearCode2}	Webber, J. REST in Practice. Hypermedia and Systems Architecture / J. Webber. – Санкт-Петербург : Питер, 2010. – 448 с. – ISBN 978-0-5968-0582-1. – Текст : непосредственный.
	\bibitem{oop3} Фримен, Э. Паттерны проектирования. Обновленное юбилейное издание. / Э. Фримен, Э. Робсон, К. Сьерра, Б. Бейтс – СПБ.: Питер, 2018. – 656с. – ISBN 978–5–496–03210–0. – Текст : непосредственный.
	\bibitem{int2} Мандел, Т. Разработка пользовательского интерфейса / Т. Мандел. – ДМК Пресс, 2019. – 420 с. – ISBN 978-5-04-195060-6. – Текст : непосредственный.
	\bibitem{int1} Кон, М. Пользовательские истории: гибкая разработка программного обеспечения. Пользовательские истории / М. Кон. – Санкт-Петербург : Диалектика, 2019. – 258 с. ~– ISBN 978-5-04-214135-5. ~– Текст : непосредственный.
	\bibitem{ddos} Блокдик, Г. DDoS Protection / Г. Блокдик. – Emereo Pty Limited, 2018. – 280 с. – ISBN 978-0-655-43937-0. – Текст : непосредственный.
	\bibitem{clearCode1} Мартин, Р. Чистый код. Создание, анализ и рефакторинг / Р. Мартин. – Санкт-Петербург : Питер, 2020. – 464 с. – ISBN 978-5-4461-0960-9. – Текст : непосредственный.
	\bibitem{test} Липаев, В. В. Тестирование компонентов и комплексов программ / В. В. Липаев. – Москва : Директ-Медиа, 2015. – 528 с. – ISBN 978-5-4475-3865-1. – Текст : непосредственный.
\end{thebibliography}
%\end{hyphenrules}